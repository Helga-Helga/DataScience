\documentclass[a4paper,14pt,russian]{extreport}
%\documentclass[tikz,14pt,russian]{standalone}
% Third parameter is needed for TOC displaying (why?)
\usepackage{../common/dsturep} % оформление по ДСТУ 3008-95
\usepackage{import}
\usepackage{standalone}
\usepackage{comment}
\usepackage{bbm}

\usepackage{tikz}
\usepackage{tikz-3dplot}
\usetikzlibrary{calc}
\usetikzlibrary{plotmarks}
\usepackage{pgfplots}

%\usepackage{scrextend}
\usepackage{changepage}
\usepackage{caption}
\usepackage{listings}
%\usepackage[title,titletoc]{appendix}
%\usepackage{appendix}
\usepackage{longtable}
%\usepackage{slashbox}
\usepackage{diagbox}
\usepackage{lscape}
\usepackage{algorithmic}
\usepackage{algorithm}

\newcommand{\indicatorof}[1]{\mathbbm{1}_{#1}}
\newcommand{\indicator}[1]{\mathbbm{1}\!\left( #1 \right)}
\newcommand{\Indicator}[1]{\mathbbm{1}\!\left\{ #1 \right\}}
\newcommand{\probability}[1]{\mathbb{P}\left( #1 \right)}
\newcommand{\Probability}[1]{\mathbb{P}\left\{ #1 \right\}}
\newcommand{\cdfof}[2]{F^{#1}\left(#2\right)}
\newcommand{\cdf}[1]{\cdfof{}{#1}}
\def\Tau{\mathrm{T}}
\newcommand{\meanof}[2]{\operatorname{M}_{#1} #2}
\newcommand{\Meanof}[2]{\meanof{#1}{\left[ #2 \right]}}
\newcommand{\mean}[1]{\meanof{}{#1}}
\newcommand{\Mean}[1]{\Meanof{}{#1}}
\newcommand{\dispersionof}[2]{\operatorname{D}_{#1} #2}
\newcommand{\Dispersionof}[2]{\dispersionof{#1}{\left[ #2 \right]}}
\newcommand{\dispersion}[1]{\dispersionof{}{#1}}
\def \mcond {\;\middle|\;}
\newcommand{\Covergencen}[1]{\xrightarrow[#1\to\infty]{}}
\newcommand{\cov}[1]{\operatorname{cov}\!\left( #1 \right)}
\DeclareMathOperator*{\argmax}{arg\,max}
\DeclareMathOperator*{\argmin}{arg\,min}

\newcommand{\drawHist}[1]{\begin{tikzpicture}[scale=1]
  \begin{axis}[ymin=-5, ymax=5, xmin=5, xmax=30, ytick=\empty,
    xmajorgrids={true},
    ylabel={Кількість точок}, ylabel near ticks,
    xlabel={час, с}]

\draw[dashed,color=gray!50] ({rel axis cs:0,0}|-{axis cs:0,0}) -- ({rel axis cs:1,0}|-{axis cs:0,0});
\addplot table [x, y, col sep=comma] {data/#1.csv};
\end{axis}
\end{tikzpicture}}

\captionsetup[subfigure]{skip=0ex} % global setting for subfigure

\usepackage{stringenc}
\usepackage{pdfescape}

\makeatletter
\renewcommand*{\UTFviii@defined}[1]{%
  \ifx#1\relax
    \begingroup
      % Remove prefix "\u8:"
      \def\x##1:{}%
      % Extract Unicode char from command name
      % (utf8.def does not support surrogates)
      \edef\x{\expandafter\x\string#1}%
      \StringEncodingConvert\x\x{utf8}{utf16be}% convert to UTF-16BE
      % Hexadecimal representation
      \EdefEscapeHex\x\x
      % Enhanced error message
      \PackageError{inputenc}{Unicode\space char\space \string#1\space
                              (U+\x)\MessageBreak
                              not\space set\space up\space
                              for\space use\space with\space LaTeX}\@eha
    \endgroup
  \else\expandafter
    #1%
  \fi
}
\makeatother
\DeclareUnicodeCharacter{00AD}{-}

\def\chapterConclusion{\section*{Висновки до розділу \arabic{chapter}}
\addcontentsline{toc}{section}{Висновки до розділу \arabic{chapter}}}

\def\male{male}
\def\female{female}


%\input{../common/minted.inc}   % оформление листингов в minted
%\bibliographystyle{../common/utf8gost705u}
%\bibliographystyle{../common/utf8gost71u}
\bibliographystyle{../common/utf8gost780u}
%\bibliographystyle{plain}

\usepackage[square,numbers,sort&compress]{natbib}
\renewcommand{\bibnumfmt}[1]{#1.\hfill} % нумерация источников в самом списке — через точку
%\renewcommand{\bibsection}

\usepackage{glossaries}
\makeglossaries

\def\passYear{2016}
\def\faculty{физико-технический институт}
\def\department{Кафедра информационной безопасности}
\def\departmentHead{Н. В. Грайворонский}
\def\kind{Дипломна робота}
\def\level{магістр}
\def\specialityCode{8.04030101}
\def\specialityTitle{Прикладная математика}
\def\theme{Наука про данные: обмен результатами и начальный анализ}
\def\gender{female}
\def\mentorGender{male}
\def\course{2}
\def\group{ФИ-41}
\def\name{Лавягина Ольга Алексеевна}
\def\mentorRank{}
\def\mentorName{Колотий Андрей Всеволодович}
\def\reviewerRank{Rank}
\def\reviewerName{Name}
\def\subject{Специальные разделы программирования}



\begin{document}

\import{1_title/}{title.tex}

\clearpage

\pagenumbering{gobble}
%\import{3_abstract/}{main.tex}

%\pagestyle{empty}
%\thispagestyle{empty}
%\tableofcontents

\clearpage
\pagenumbering{arabic}
\pagestyle{fancy}
\setcounter{page}{2}

\clearpage

\chapter{Задание}
\begin{itemize}
\item Зарегистрироваться на сайте GitHub, создать репозиторий, добавить в репозиторий код и данные из лабораторной работы №1, продемонстрировать навыки работы с системой контроля версий git на работе с проэктом GitHub;
\item создать веб-приложение с использованием модуля Spyre, которое позволит:
\begin{itemize}
\item выбрать часовой ряд VCI, TCI, VHI для наборы данных из лабораторной работы 1 (выпадающий список);
\item выбрать область, для которой будет выполняться анализ (выпадающий список);
\item указать интервал недель, за которые отбираются данные;
\item создать несколько вкладок для отображения таблицы с данными на графике хода индексов;
\end{itemize}
\item код разработанного приложения добавить к созданному репозиторию.
\end{itemize}

\chapter{Листинг кода}
\lstset{inputencoding=utf8, extendedchars=\true}
\lstinputlisting[language=C++,
                 basicstyle=\ttfamily\scriptsize]{../lab2.py}

\chapter{Пояснение}
Класс SimpleApp наследует сервер. В работе было сделано веб-приложение, которое содержит 3 вкладки (Plot, Table и Drought), то есть отображает таблицу, график и HTML (текст). Для этого были переопределены методы getData, getPlot и getHTML. Метод getData получает и генерирует данные, которые будут отображаться в таблице. Так же как getPlot и getHTML, он принимает на вход агрумент params, который является словарём, содержащим все входные переменные. Метод getData возвращает фрэйм с данными (pandas DataFrame), метод getPlot --- график, а метод getHTML --- строку.

Данные, которые необходимо вывести генерируются в данных методах на основе входных данных (inputs), которые указывабтся в полях ввода. В данной работе было использовано 8 таких полей с типами <<dropbox>> --- выпадающий список (выбор индекса и области), <<text>> --- текст (выбор года) и <<slider>> --- ползунок (выбор диапазона недель, значений индекса VHI, а также минимального процента области, где данный индекс имеет значение меньше 15).

Результатами (выходами, outputs) являются <<plot>>, <<table>> и <<html>>. Все результаты загружаются на страницу по умолчанию. Приложение требует множественный вывод, поэтому каждый результат находится на отдельной вкладке. 

В работе отображаются значения индексов VHI, THI и VHI для выбранной области за определённый период в виде таблицы и графика, каждый на своей вкладке. Также на вкладке Drought отображются года, когда индекс VHI и процент области с индексом VHI были в пределах выбранных значений.

\chapter*{Выводы}
\addcontentsline{toc}{chapter}{Выводы}



\end{document}
